\section{Ferramentas usadas}
Este projecto foi executado usando-se \textit{Python} como linguagem de referência. Contudo, para facilitar o nosso trabalho, usamos algumas bibliotecas \textit{open source}:
\begin{itemize}
\item \textit{Pandas}: usado para ler os dados dos \textit{datasets} originalmente em \textit{csv} para matrizes de \textit{numpy}.
\item \textit{NumPy}: usado para fazer cálculos matriciais mais facilmente.
\item \textit{Matplotlib}: usado para criação facilitada de gráficos e figuras.
\item \textit{scikit-learn}: usado para estudar alguns dos algoritmos de aprendizagem automática descritos neste relatório.
\item \textit{Keras} e \textit{TensorFlow}: usados para estudar alguns dos algoritmos de aprendizagem automática descritos neste relatório.
\item \textit{Pickle}: usado para armazenar e carregar os modelos treinados em ficheiros binários.
\end{itemize}
