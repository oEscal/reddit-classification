\section{Introdução}
Aprendizagem Automática é uma área das ciências da computação com cada vez mais destaque nos dias que correm. Desde a sua utilização em campos como a saúde, sistemas bancários, \textit{self driving cars}, nas aplicações de telemóvel do utilizador comum, entre outros, não é surpresa para ninguém que se tornou uma realidade sem a qual a sociedade em que vivemos seria muito diferente do que é hoje. 

Este trabalho, proposto pela professora Petia Georgieva, do Departamento de Electrónica Telecomunicações e Informática da Universidade de Aveiro, teve o intuito de consolidar e instigar a utilização e pesquisa dos conceitos apresentados na disciplina de Tópicos de Aprendizagem Automática.
Desta forma, estudamos o comportamento dos algoritmos \textbf{Redes Neuronais}, \textbf{Redes de \textit{Bayes}} e \textbf{Regressão Logística} no âmbito de um trabalho de classificação de texto, mais especificamente classificação de textos em \textit{subreddits} obtidos da rede social \textit{\textbf{Reddit}} \footnote{\textbf{\textit{Subreddits}} consistem em comunidades do \textbf{\textit{Reddit}}, onde os utilizadores associados falam entre si sobre um tema em concreto, associado a esse mesmo \textbf{\textit{subreddit}}}. Esta rede social é caracterizada por permitir a interacção publica e livre entre os seus utilizadores sobre os mais variados temas. De forma a que estas discussões e divulgação de informação sejam feitas de forma organizada e explicita, os temas são usualmente agrupados em \textit{subreddits} que são, no fundo, grupos em que qualquer utilizador da rede social se pode conectar e discutir, ou apenas observar uma discussão, sobre esse assunto em específico.     % TODO -> se calhar tira se o footnote para não ser tão redundante?

A razão de termos escolhido este tema, foi que ia de encontro a uma das \textit{features} que o projecto de informática dos dois autores requeria: analisar \textit{tweets} e obter qual o tema principal que era discutido nos mesmos. Para mais informações sobre este projecto é favor consultar a página \cite{informatics_project}. Sendo que, tal como no estudo discutido neste relatório, é suposto criarmos um modelo que tenha uma boa performance a detectar qual o \textit{subreddit} associado a uma postagem e, sendo assim, tendo em conta o conceito de \textit{subreddit}, ligar qual o "tema de conversa" de uma determinada postagem \footnote{Neste trabalho, tal como perceptível nesta introdução, é tida como premissa que um \textit{subreddit} representa, em ultima instância, o "tema de conversa" associado ás postagens feitas nesse \textit{subreddit}.} a ela mesma, pretendemos encontrar qual o modelo que tem a melhor performance em detectar os temas duma dada postagem, de forma a utiliza-lo para fazer essa mesma tarefa em relação aos \textit{tweets} associados ao nosso projecto de informática.

Além da implementação e estudo dos algoritmos enumerados anteriormente, foi também feita a implementação e estudo aprofundado de pré-processamento e extracção de \textit{feature} de texto.
